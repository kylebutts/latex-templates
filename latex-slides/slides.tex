\documentclass[aspectratio=43]{beamer}
% \documentclass[aspectratio=169]{beamer}

% Title --------------------------------------------
\title{What is the Effect of X on Y?}
\date{\today}
\author{Kyle Butts}

% xcolor and define colors -------------------------
\usepackage{xcolor}
% https://www.materialpalette.com/colors
\definecolor{red}{HTML}{c62828}
\definecolor{orange}{HTML}{ef6c00}
\definecolor{green}{HTML}{2e7d32}
\definecolor{blue}{HTML}{1565c0}
\definecolor{purple}{HTML}{283593}
\definecolor{maroon}{HTML}{AF3335}
\definecolor{teal}{HTML}{00695c}
\definecolor{bluegrey}{HTML}{455a64}


% CU Boulder colors ---------------------------------
\definecolor{buff-gold}{HTML}{CFB87C}
\definecolor{buff-grey}{HTML}{565A5C}
\definecolor{buff-lightgrey}{HTML}{A2A4A3}
\definecolor{buff-black}{HTML}{000000}

% Beamer Options -------------------------------------

% Background
\definecolor{mybackground}{HTML}{ECECEC}
\setbeamercolor{background canvas}{bg= mybackground}

% \alert
\setbeamercolor{alerted text}{fg= buff-gold!80!black}

% Frame title
\setbeamercolor{frametitle}{bg= buff-black}
\setbeamercolor{title}{fg= buff-grey}

% Button 
\setbeamercolor{button}{bg= buff-gold}

% Block
\metroset{block=fill}

% Enumitem ------------------------------------------
% Allow to remove indent w/ \begin{itemize}[leftmargin= *]
\usepackage{enumitem}
\setlist[itemize]{label= \textbullet}

% \begin{columns} -----------------------------------
\usepackage{multicol}

% Math Font -----------------------------------------
\usepackage[libertine]{newtxmath}


% \imageframe{img_name} -----------------------------
% from https://github.com/mattblackwell/cousteau
\newcommand{\imageframe}[1]{%
    \begin{frame}[plain]
        \begin{tikzpicture}[remember picture, overlay]
            \node[at = (current page.center), xshift = 0cm] (cover) {%
                \includegraphics[keepaspectratio, width=\paperwidth, height=\paperheight]{#1}
            };
        \end{tikzpicture}
    \end{frame}%
}

% Table of Contents with Sections
\setbeamerfont{myTOC}{series=\bfseries, size=\Large}
\AtBeginSection[]{\frame{\frametitle{Outline}%
                  \usebeamerfont{myTOC}\tableofcontents[current]}}

% Set-up Bibliography ------------------------------
\addbibresource{references.bib}

\begin{document}

% ------------------------------------------------------------------------------
\begin{frame}
\maketitle

% \vspace{2.5mm}
% {\footnotesize $^*$ A bit of extra info here. Add an asterich to title or author}
\end{frame}
% ------------------------------------------------------------------------------

% ------------------------------------------------------------------------------
\section{Common Items}
% ------------------------------------------------------------------------------

\begin{transitionframe}
Transitioning Sentence
\end{transitionframe}

\begin{frame}{Components}{Bullet Points \& Button}\label{main1}
    This section highlights commonly used components and their theming

    \begin{itemize}
        \item Can emphasize with \alert{the alert command}
        
        \begin{itemize}
            \item This allows you to draw attention to specific words/phrases
        \end{itemize}
        
        \item To include things in appendix, you must first label the slide and the appendix slide and then include a hyperlink:
        
        \vspace{5mm}
        \hyperlink{appendix1}{\beamergotobutton{Appendix}}
    \end{itemize}
\end{frame}

\begin{frame}{Components}{Numbered Lists}
    You can also use numbered items that look a bit more professional

    \begin{enumerate}
        \item Pretty good
        
        \item To include things in appendix
    \end{enumerate}
\end{frame}

\begin{frame}{Components}{Citations}
    Topic 1: Spatial Frictions
    \begin{citecolor}
        [\citet{Fajgelbaum_Morales_Serrato_Zidar_2018}, \citet{Hsieh_Moretti_2019}, and \citet{Moretti_2011}]
    \end{citecolor}

    \vspace{5mm}
    Topic 2: Blah 
    \begin{citecolor}
        [\citet{Suárez_Serrato_Zidar_2016}]
    \end{citecolor}
\end{frame}

\begin{frame}{Components}{Blocks}
    \begin{block}{Regression Specification}
        The main specificaiton is as follows: 

        \[
            y_{it} = X_{it} \beta + \mu_i + \varepsilon_{it}
        \]
    \end{block}
\end{frame}

\begin{frame}{Components}{Colors}
  \navy{Test sentence with \textbackslash navy\{...\}}

  \purple{Test sentence with \textbackslash purple\{...\}}

  \kelly{Test sentence with \textbackslash kelly\{...\}}

  \ruby{Test sentence with \textbackslash ruby\{...\}}

  \alice{Test sentence with \textbackslash alice\{...\}}

  \daisy{Test sentence with \textbackslash daisy\{...\}}

  \coral{Test sentence with \textbackslash coral\{...\}}

  \cranberry{Test sentence with \textbackslash color\{cranbery\}}

  \slate{Test sentence with \textbackslash color\{slate\}}

  \jet{Test sentence with \textbackslash color\{jet\}}

  \asher{Test sentence with \textbackslash color\{asher\}}
\end{frame}

\begin{frame}{Components}{Colored Boxes}
  \bgNavy{Test word} with \textbackslash bgNavy\{...\}

  \vspace{2.5mm}
  \bgPurple{Test word} with \textbackslash bgPurple\{...\}

  \vspace{2.5mm}
  \bgKelly{Test word} with \textbackslash bgKelly\{...\}

  \vspace{2.5mm}
  \bgRuby{Test word} with \textbackslash bgRuby\{...\}

  \vspace{2.5mm}
  \bgAlice{Test word} with \textbackslash bgAlice\{...\}

  \vspace{2.5mm}
  \bgDaisy{Test word} with \textbackslash bgDaisy\{...\}

  \vspace{2.5mm}
  \bgCoral{Test word} with \textbackslash bgCoral\{...\}

  \vspace{2.5mm}
  \bgCranberry{Test word} with \textbackslash bgCranberry\{...\}
\end{frame}

\begin{frame}{Components}{Two Columns}
    \begin{columns}[T]
    \vspace{0pt}
    \begin{column}{.50\textwidth}
        \vspace{0pt}
        {\color{accent}\rule{\linewidth}{2pt}}
        Column 1

        \begin{enumerate}
            \item Bullet points for this column that can go over lines
            \item b
            \item c
        \end{enumerate}

        \vspace*{50mm} % Ensures columns are up top; delete if you want columns centered on page
    \end{column}
    
    \hfill
    
    \begin{column}{.50\textwidth}
        {\color{accent}\rule{\linewidth}{2pt}}
        Column 2

        \begin{itemize}
            \item a
            \item b
            \item c
        \end{itemize}
    \end{column}
    \end{columns}
\end{frame}

\begin{frame}{Components}{Two Columns with Figure}
    \begin{columns}[T]
    \vspace{0pt}
    \begin{column}{.60\textwidth}
        \includegraphics[width=\textwidth]{img/kanagawa.jpg}

        \vspace*{50mm} % Ensures columns are up top; delete if you want columns centered on page
    \end{column}
    
    \hfill
    
    \begin{column}{.40\textwidth}
        \begin{itemize}
            \item A point about the figure that is potentially important.
            \item Another point about the figure that is also potentially important.
        \end{itemize}
    \end{column}
    \end{columns}
\end{frame}

% ------------------------------------------------------------------------------
\section{Table}
% ------------------------------------------------------------------------------

\begin{frame}
    
% Adjust Font Size to make table fit   
\begin{table}[!htbp]
    \caption{Regression Results} 
    \label{}
    
    \begin{adjustbox}{max width = \textwidth, width = 0.5\textwidth, center}
        \begin{threeparttable}
            \begin{tabular}{@{} l *{2}{r} @{}} 
                \toprule
                & \multicolumn{2}{c}{\textit{Dependent variable: Overall Rating}} \\ 
                \cline{2-3} \\
                & (1) & (2)\\ 
                \midrule
                
                \only<1>{
                    Handling of Complaints & 0.692$^{***}$& 0.682$^{***}$ \\ 
                    &  (0.149) & (0.129) \\
                }
                \only<2>{
                    \marktopleft{ex1}Handling of Complaints & 0.692$^{***}$& 0.682$^{***}$ \\ 
                    &  (0.149) & (0.129) \markbottomright{ex1} \\
                }
                No Special Privileges & -0.104 & $-$0.103  \\ 
                & (0.135) & (0.129) \\
                Opportunity to Learn & 0.249 & 0.238$^{*}$ \\ 
                & (0.160) & (0.139) \\
                Performance-Based Raises & -0.033 &  \\ 
                & (0.202) & \\
                Too Critical & 0.015 &  \\ 
                & (0.147) & \\
                Advancement & 11.011 & 11.258 \\ 
                & (11.704) & (7.318) \\
                
                \midrule 
                Observations & 30 & 30 \\ 
                R$^{2}$ & 0.715 & 0.715 \\ 
                \bottomrule

            \end{tabular} 
            % Notes 
            \begin{tablenotes}
                \item \textit{Notes.} $^{*} p<0.1$; $^{**} p<0.05$; $^{***} p<0.01$.
            \end{tablenotes}
        \end{threeparttable}
    \end{adjustbox}
\end{table}

{\small
Use \textbackslash marktopleft\{name\} and \textbackslash markbottomright\{name\} to create box.
}
\end{frame}

% ------------------------------------------------------------------------------
\section{Figures}
% ------------------------------------------------------------------------------


% 4:3 ratio for figures
% or else it will have white space
\imageframe{img/kanagawa.jpg}

\begin{frame}{Figure}{Full-size Figures}
    You can use the command \textbackslash imageframe\{img-path\} and it will create a full-frame of a picture. 
    
    \begin{itemize}
        \item Ideally, your figure is the same aspect as the frame $4:3$ or $16:9$ or else there will be white space in one of the directions.
    \end{itemize}
\end{frame}

\begin{frame}{Figure}{}
    \begin{center}
        \resizebox{0.9\columnwidth}{!}{%
            \includegraphics{img/kanagawa.jpg}
        }
    \end{center}
\end{frame}

% ------------------------------------------------------------------------------
\begin{frame}[allowframebreaks]{References}
    \printbibliography
\end{frame}
% ------------------------------------------------------------------------------


% ------------------------------------------------------------------------------
\appendix
% ------------------------------------------------------------------------------

\begin{frame}{Appendix Slide}\label{appendix1}

\begin{table}
    \caption{Summary Statistics}
    \label{appendix_summ_stat}
    \begin{adjustbox}{max width = \textwidth, width = 0.8 \textwidth, center}
        \begin{threeparttable}
            \begin{tabular}{@{} lccccccc @{}} 
                \toprule
                Statistic & \multicolumn{1}{c}{N} & \multicolumn{1}{c}{Mean} & \multicolumn{1}{c}{St. Dev.} & \multicolumn{1}{c}{Min} & \multicolumn{1}{c}{Pctl(25)} & \multicolumn{1}{c}{Pctl(75)} & \multicolumn{1}{c}{Max} \\ 
                \hline \\[-1.8ex] 
                rating & 30 & 64.633 & 12.173 & 40 & 58.8 & 71.8 & 85 \\ 
                complaints & 30 & 66.600 & 13.315 & 37 & 58.5 & 77 & 90 \\ 
                privileges & 30 & 53.133 & 12.235 & 30 & 45 & 62.5 & 83 \\ 
                learning & 30 & 56.367 & 11.737 & 34 & 47 & 66.8 & 75 \\ 
                raises & 30 & 64.633 & 10.397 & 43 & 58.2 & 71 & 88 \\ 
                critical & 30 & 74.767 & 9.895 & 49 & 69.2 & 80 & 92 \\ 
                advance & 30 & 42.933 & 10.289 & 25 & 35 & 47.8 & 72 \\ 
                \bottomrule
            \end{tabular} 
            % Notes 
            \begin{tablenotes}
                \item \textit{Notes.} Using R base dataframe attitude.
            \end{tablenotes}
        \end{threeparttable}
    \end{adjustbox}
\end{table}
    
    \hyperlink{main1}{\beamergotobutton{Back to Main}}
\end{frame}
\end{document}
