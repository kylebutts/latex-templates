\documentclass[12pt]{article}
% Should copy over the .sty files from latex-article
\usepackage{../latex-article/paper}
\usepackage{../latex-article/math}
\usepackage{referee-response}

% Conditionally display thoughts (hide by switching to `\boolfalse`)
\booltrue{INCLUDECOMMENTS}
\newcommand{\kyle}[1]{\coauthorComment[Kyle]{#1}}

% Reference main document ------------------------------------------------------
% Note: I compile to `auxiliary/` to avoid cluttering the main directory. You might neeed to delete `auxiliary/`
% \usepackage{xr}
% \externaldocument[main-]{auxiliary/article}

\begin{document}

% Title ------------------------------------------------------------------------
\begin{center}
    {\large\bf Paper Title}
    
    \emph{Authors' Responses to Referee Comments}
    
    \emph{Journal} (\texttt{Review Number})
\end{center}
% ------------------------------------------------------------------------------

\noindent Thank you for the additional suggestions. The reviewer's comment is marked with a left border. Our responses follow below each comment.

\section*{Summary of Revisions}

The command \texttt{\textbackslash NewRef\{Name\}\{Abbreviation\}} will create a \texttt{\textbackslash section*} for that referee's comments

% ------------------------------------------------------------------------------
\NewRef{Referee 1}{R1}
% ------------------------------------------------------------------------------

\begin{refcomment}
  Full text of the referee's comment. It will automatically create a \texttt{\textbackslash subsection*} with the comment number.
\end{refcomment}

\kyle{
  Thoughts / plan of action. These can be hidden by toggling \texttt{INCLUDETODOS} to \texttt{\textbackslash boolfalse}. 
}

Response

\begin{refcomment}
  Comment 
\end{refcomment}

Response



% ------------------------------------------------------------------------------
\NewRef{Referee 2}{R2}
% ------------------------------------------------------------------------------





\end{document}
